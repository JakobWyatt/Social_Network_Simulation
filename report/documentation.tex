\documentclass{article}

\usepackage{microtype}

\title{COMP1002 Assignment\\Documentation}
\author{Jakob Wyatt\\19477143}

\begin{document}
\maketitle
\pagebreak
\section{Overview}
This program is designed to simulate a social network.
The user interface is as specified in the assignment brief,
with additional functionality including graphical representation
of the socical network in interactive mode, and interactive liking/unliking 
of posts.\\
There exists only one post at a time, with a new post being loaded when the current
post has not had any activity in the last timestep. The original poster always
likes their own post. A user can only like a post once.\\
The simulation consists
of timesteps, with a function \texttt{update()} to move between timesteps.\\
The social network consists of a set of users that may follow eachother,
which has been represented in code as a directed graph. Users may not follow themselves,
or follow eachother more than once.\\
The update algorithm works as follows:
\begin{enumerate}
    \item Check that there exists some users that have liked the post in the previous timestep.
            If there are none, the update ends and the next post is loaded.
    \item Iterate through all users who liked the post in the previous timestep.
            Each of their followers is 'exposed' to the post, and have a chance of liking the post.
            This chance is sampled from a Bernoulli distribution with probability
            $\mathit{clamp}(\mathit{prob\_like} \times \mathit{clickbait\_factor}, 0, 1)$.
    \item If a user likes a post in the current timestep, they have a chance of following the
            original poster. This is sampled using the same technique as above, with global probability
            $\mathit{prob\_foll}$.g
\end{enumerate}
Note that in the above algorithm, if a user does not like a post, they may potentially
be exposed to it later via a different friend. This behaviour is intentional, as it incentivises
a highly connected network.

\section{Justification}
In any program, whenever an ADT or algorithm is selected for use, this use must be justified.
Choice of an ADT must help enhance clarity, make implementation of an algorithm easier,
and have good time and space complexity.

The user network uses the \texttt{DSADirectedGraph} class to represent follows between users
as edges. This helps to enhance clarity, as follows are inherently directional in nature.

Operations that must be performed within this directed graph are find/add/remove verticies,
and find/add/remove edges. It should also be noted that within the context of a social network,
there only exists a maximum of one directed edge between any 2 nodes. Duplicate nodes also will not
exist.
To store verticies and edges within this directed graph, the \texttt{DSALinkedList} class was originally
used to store this data. However, this ended up being rather inefficient, with a time complexity of
$O\left(V\right)$ for vertex find/remove and $O\left(E + V\right)$ for edge find/remove.
Although vertex add and edge add are theoretically $O\left(1\right)$, in practice they had a time complexity
equal to vertex find and edge find, as it had to be checked that the vertex/edge did not already exist in the network.

To improve the performance of these operations, a DSAHashTable was used to store the verticies and edges instead.
This gives $O\left(1\right)$
performance for find, insert, and remove operations. Despite the greater overhead, in practice it was found to outperform
the linked list implementation at a graph size of 20, approximately doubling the speed when the number of verticies was 50.

\section{Generated Documentation}
\end{document}
