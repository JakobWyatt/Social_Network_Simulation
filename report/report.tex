\documentclass{article}

\usepackage{microtype}

\title{Performance and Simulation of Social Networks}
\author{Jakob Wyatt\\19477143}

\begin{document}
\maketitle
\section{Abstract}
\pagebreak
\section{Background}
This report focuses on simulating simple social media networks,
and evaluates the effectiveness of a network given parameters
of that network.\\
The social network consists of a set of users that may follow eachother,
which has been represented in code as a directed graph. Users may not follow themselves,
or follow eachother more than once.\\
There exists only one post at a time, with a new post being loaded when the current
post has not had any activity in the last timestep. The original poster always
likes their own post. A user can only like a post once.\\
The simulation consists
of timesteps, with a function \texttt{update()} to move between timesteps.
The update algorithm works as follows:
\begin{enumerate}
    \item Check that there exists some users that have liked the post in the previous timestep.
            If there are none, the update ends and the next post is loaded.
    \item Iterate through all users who liked the post in the previous timestep.
            Each of their followers is 'exposed' to the post, and have a chance of liking the post.
            This chance is sampled from a Bernoulli distribution with probability
            $\mathit{clamp}\left(\mathit{prob\_like} \times \mathit{clickbait\_factor}, 0, 1\right)$.
    \item If a user likes a post in the current timestep, they have a chance of following the
            original poster. This is sampled using the same technique as above, with global probability
            $\mathit{prob\_foll}$.
\end{enumerate}
Note that in the above algorithm, if a user does not like a post, they may potentially
be exposed to it later via a different friend. This behaviour is intentional, as it incentivises
a highly connected network.

Parameters that will be tracked before, during, and after the simulation include:
\begin{itemize}
    \item Clustering Coefficient
    \item Density (Proportion of follows in the network relative to the maximum possible)
    \item Average and s.d. of followers per user
\end{itemize}

Some parameters that will be varied in the creation of the social network include:
\begin{itemize}
    \item Probability of liking a post
    \item Probability of following a user
    \item Number of users
    \item Selection of Clickbait Factor
\end{itemize}

The main metric of a social networks performance, from a monetary perspective,
is likes per person per post, which will be measured during simulation of the network.
\section{Methodology}

\section{Results}
\section{Conclusion and Future Work}
\end{document}
